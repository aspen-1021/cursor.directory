\section{Research Background (研究背景)}
\label{sec:research_background}

\subsection{Evolution of 3D Architectural Visualization Technology}
\label{subsec:evolution_3d_visualization}

The landscape of architectural visualization has undergone a profound transformation over the past three decades, evolving from static two-dimensional drawings to immersive, interactive three-dimensional environments. This evolution has been driven by rapid advances in computer graphics, rendering technologies, and human-computer interaction paradigms, fundamentally reshaping how architects, designers, and end-users conceptualize, communicate, and experience architectural spaces.

\subsubsection{Historical Development and Technological Milestones}
\label{subsubsec:historical_development}

The journey began in the 1980s with the introduction of Computer-Aided Design (CAD) systems, which primarily focused on technical documentation and geometric precision rather than visual communication. Early systems such as AutoCAD revolutionized the drafting process but offered limited capabilities for realistic visualization. The 1990s marked a significant turning point with the emergence of 3D modeling software like 3ds Max and Maya, originally developed for the entertainment industry but rapidly adopted by architectural professionals seeking more sophisticated visualization tools.

The advent of Building Information Modeling (BIM) in the early 2000s represented a paradigm shift from mere visualization to comprehensive digital building lifecycle management. Platforms such as Autodesk Revit, ArchiCAD, and Bentley MicroStation integrated 3D modeling with parametric design capabilities, enabling architects to create intelligent building models that contained not only geometric information but also semantic data about materials, systems, and performance characteristics.

\subsubsection{Real-Time Rendering and Interactive Visualization}
\label{subsubsec:realtime_rendering}

The introduction of real-time rendering engines marked another revolutionary milestone in architectural visualization. Game engine technologies, particularly Unreal Engine and Unity, began finding applications in architectural contexts around 2010, enabling interactive walkthroughs and real-time design exploration. These platforms democratized high-quality visualization by reducing the computational time required for photorealistic rendering from hours to milliseconds, making interactive design review sessions feasible for the first time.

The integration of physically-based rendering (PBR) techniques has further enhanced the realism and accuracy of 3D architectural models. Modern rendering engines can now simulate complex lighting phenomena, material properties, and environmental conditions with unprecedented fidelity, creating visual experiences that closely approximate real-world conditions. This technological advancement has been crucial in bridging the gap between conceptual design and user perception of built environments.

\subsubsection{Immersive Technologies and Extended Reality}
\label{subsubsec:immersive_technologies}

The emergence of Virtual Reality (VR) and Augmented Reality (AR) technologies has opened new frontiers in architectural visualization. VR headsets such as the Oculus Rift, HTC Vive, and more recently, the Meta Quest series, have enabled fully immersive architectural experiences that allow users to navigate and interact with spaces at human scale. These technologies have proven particularly valuable for design validation, client presentations, and user experience testing before construction begins.

Augmented Reality has complemented VR by enabling the overlay of digital architectural information onto real-world environments. AR applications allow architects and clients to visualize proposed designs in their actual contexts, facilitating better decision-making regarding scale, proportion, and environmental integration. The development of mobile AR platforms, particularly Apple's ARKit and Google's ARCore, has made these technologies more accessible to a broader user base.

\subsection{Current State of User Engagement in Digital Architectural Environments}
\label{subsec:current_user_engagement}

\subsubsection{Evolving User Expectations and Interaction Paradigms}
\label{subsubsec:evolving_expectations}

Contemporary users of architectural visualization tools exhibit increasingly sophisticated expectations regarding interactivity, responsiveness, and visual fidelity. The widespread adoption of high-resolution displays, touch interfaces, and gesture-based controls in consumer electronics has raised the bar for professional architectural applications. Users now expect intuitive navigation, real-time feedback, and seamless integration across multiple devices and platforms.

The democratization of 3D technologies has also expanded the user base beyond traditional architectural professionals. Real estate developers, marketing teams, facilities managers, and end-users now regularly interact with 3D architectural models for various purposes, from property marketing to space planning. This diversification has created new challenges in designing user interfaces that accommodate varying levels of technical expertise and domain knowledge.

\subsubsection{Multi-Modal Interaction and Accessibility}
\label{subsubsec:multimodal_interaction}

Modern 3D architectural visualization platforms increasingly support multi-modal interaction paradigms, combining traditional mouse and keyboard inputs with touch gestures, voice commands, and spatial controllers. The integration of haptic feedback technologies has begun to add tactile dimensions to digital architectural experiences, enabling users to ``feel'' material properties and spatial relationships.

Accessibility considerations have gained prominence as organizations recognize the importance of inclusive design in digital architectural tools. Screen reader compatibility, alternative input methods for users with mobility impairments, and visual accommodations for users with color vision deficiencies are becoming standard requirements rather than optional features.

\subsubsection{Cloud Computing and Collaborative Visualization}
\label{subsubsec:cloud_collaborative}

The shift toward cloud-based architectural visualization platforms has transformed the collaborative aspects of design review and decision-making processes. Platforms such as Autodesk Construction Cloud, Bentley iTwin, and various web-based viewers enable real-time collaboration among geographically distributed teams. These technologies have become particularly critical following the COVID-19 pandemic, which accelerated the adoption of remote work practices in the architecture, engineering, and construction (AEC) industry.

Cloud computing has also enabled more sophisticated analytics and data collection capabilities. Organizations can now track user interactions with 3D models across multiple sessions and devices, creating opportunities for detailed behavioral analysis and user experience optimization. However, this increased data collection capability has also raised important questions about privacy, data ownership, and analytical methodology.

\subsection{The Gap Between Visual Representation and User Preferences}
\label{subsec:gap_visual_preferences}

\subsubsection{Limitations of Current Visualization Approaches}
\label{subsubsec:visualization_limitations}

Despite significant technological advances, a substantial gap persists between the visual information presented in 3D architectural models and users' actual preferences and decision-making processes. Traditional visualization approaches tend to prioritize photorealistic appearance over user experience metrics, often resulting in visually impressive but analytically limited representations.

Current 3D architectural models typically emphasize aesthetic qualities such as lighting, materials, and spatial composition while providing limited insight into how users actually perceive, navigate, and evaluate these spaces. This emphasis on visual fidelity over behavioral understanding represents a significant missed opportunity for evidence-based design optimization.

\subsubsection{Challenges in Measuring and Interpreting User Responses}
\label{subsubsec:measuring_user_responses}

The measurement of user engagement and preference in 3D architectural environments presents unique methodological challenges. Traditional user experience research methods, developed primarily for two-dimensional interfaces, often prove inadequate for three-dimensional spatial environments where users must navigate, orient themselves, and process complex visual information simultaneously.

Existing analytics tools typically capture basic interaction metrics such as click counts, session duration, and navigation paths, but fail to provide deeper insights into visual attention patterns, emotional responses, or preference formation mechanisms. The translation of these limited metrics into actionable design insights remains a significant challenge for both researchers and practitioners.

\subsubsection{Disconnect Between Design Intent and User Reception}
\label{subsubsec:design_intent_disconnect}

A fundamental disconnect often exists between architects' design intentions and users' actual experiences of digital architectural spaces. Designers may emphasize certain spatial qualities or material characteristics that users overlook or interpret differently than intended. This disconnect is particularly pronounced when cultural, demographic, or experiential differences exist between design teams and end-users.

The lack of systematic feedback mechanisms for capturing and analyzing user responses to specific design elements compounds this problem. Without robust methodologies for correlating visual features with user preferences, designers are often forced to rely on intuition, anecdotal feedback, or limited focus group data when making critical design decisions.

\subsubsection{Opportunities for Data-Driven Design Optimization}
\label{subsubsec:data_driven_opportunities}

The convergence of advanced computer vision techniques, machine learning algorithms, and comprehensive user tracking capabilities creates unprecedented opportunities for evidence-based design optimization. Pixel-level analysis of user visual attention patterns, combined with preference modeling and automated design generation, could potentially revolutionize how architectural products are developed and refined.

However, realizing this potential requires sophisticated analytical frameworks that can effectively bridge the gap between low-level visual data and high-level design insights. The development of such frameworks represents both a significant technical challenge and a transformative opportunity for the architectural visualization industry.

\subsubsection{Implications for Product Development and Innovation}
\label{subsubsec:product_development_implications}

The growing importance of user experience in architectural product development has created new demands for quantitative methods that can predict and optimize user engagement. Real estate developers, furniture manufacturers, and architectural service providers are increasingly seeking data-driven approaches to validate design decisions and optimize their offerings for target markets.

This trend toward evidence-based design has created a market need for analytical tools and methodologies that can effectively translate user interaction data into actionable business insights. Organizations that can successfully develop and deploy these capabilities are likely to gain significant competitive advantages in an increasingly data-driven marketplace.

The integration of user preference analysis into the design process also presents opportunities for more personalized and adaptive architectural experiences. As machine learning technologies mature, it may become feasible to create architectural visualization systems that automatically adapt their presentation and interaction modalities based on individual user characteristics and preferences.

\subsection{Research Foundation and Motivation}
\label{subsec:research_foundation}

This research background establishes the foundation for investigating how pixel-level analysis of user interactions with 3D architectural models can be systematically transformed into preference insights and product optimization strategies, addressing a critical gap in current visualization and design methodologies. The convergence of technological capability, market demand, and methodological opportunity creates a compelling case for developing comprehensive frameworks that can bridge the gap between visual data analytics and architectural design optimization.